\section{Introduction}

\slide{Introduction}{}{
  This is the first slide.
  \\~\\
  これは自分の Beamer のテンプレートです。
  \\~\\
  空行は \texttt{\textbackslash\textbackslash\textasciitilde\textbackslash\textbackslash} です
}

\section{Body}

\subsection{hoge}

\slide{Itemize}{}{
  Body Section

  wideitemize - itemize
  \begin{wideitemize} % 間がちょっと広い
  \item Coq
  \item Haskell
  \item ML
    \begin{itemize}
    \item OCaml
    \item SML
    \item SML\#
    \end{itemize}
  \item Scala
  \end{wideitemize}
}

\slide{日本語}{}{
  これは日本語です。

  \begin{align}
    f(x) & = x^2 \\
    g(x) & = 2x
  \end{align}
}

\subsection{block 系}

\slide{Block}{block,exampleblock,alertblock}{
  \begin{block}{block}
    \tt{block}
  \end{block}

  \begin{exampleblock}{exampleblock}
    \tt{exampleblock}

    \coqinline{Theorem modus\_ponens : forall P Q, P -> (P -> Q) -> Q}
  \end{exampleblock}

  \begin{alertblock}{alertlock}
    \tt{alertblock}
  \end{alertblock}
}

\slide{Theorem}{theorem,corollary, fact, lemma}{
  \begin{theorem}[すごい定理]
    For any $a \in A$, $a$ はなんかすごい性質を満たす
  \end{theorem}

  \begin{corollary}[すごい系]
    For any $a \in A$, $a$ はなんかすごい性質を満たす
  \end{corollary}

  \begin{fact}[すごい事実]
    For any $a \in A$, $a$ はなんかすごい性質を満たす
  \end{fact}

  \begin{lemma}[すごい補題]
    For any $a \in A$, $a$ はなんかすごい性質を満たす
  \end{lemma}
}

\slide{Definition}{}{
  \begin{definition}[基本的な定義]
    ほげ
  \end{definition}

  \begin{example}[例っぽい]
    「なになにっぽい」ってなんなんでしょうか
  \end{example}

  \begin{proof}[すごい定理の証明です]
    明らか
  \end{proof}

  \begin{proof}
    すごい定理から明らか
  \end{proof}
}

\slide{Inference rules}{}{
  from bcprules.sty placed at \texttt{./sty/}

  \infrule[Nil]{
  }{
    nil : list A
  }

  \infrule[Cons]{
    a : A
    \andalso
    l : list A
  }{
    a :: l : list A
  }
}

\subsection{Source Codes}

\slide{Source code}{Coq}{
  \lstinputlisting[language=Coq,style=default]{./src/coq.v}
}

\slide{Source code}{Erlang}{
  \lstinputlisting[language=Erlang,style=default,caption=Erlang]{./src/erlang.erl}
}

\begin{frame}[fragile]{lstlisting}{}

  requires \texttt{fragile} option

  \begin{lstlisting}[style=default,caption={[JS]JavaScript}]
    function() {
      hoge;
      poyo;
      return this;
    }
  \end{lstlisting}
\end{frame}

\subsection{fuga}

\slide{Image}{}{
  \begin{figure}
    \includegraphics[width=5cm]{./img/psg.png}
  \end{figure}

  PSG のロゴ\footnote{\url{http://www.psg.cs.titech.ac.jp}}
}

\section{Conclusion}

\slide{Conclusion}{}{
  This is the last slide.
}
